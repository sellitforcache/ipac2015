\documentclass[a4paper,
               %boxit,
               %titlepage,   % separate title page
               %refpage      % separate references
              ]{jacow}

\makeatletter%                           % test for XeTeX where the sequence is by default eps-> pdf, jpg, png, pdf, ...
\ifboolexpr{bool{xetex}}                 % and the JACoW template provides JACpic2v3.eps and JACpic2v3.jpg which might generates errors
 {\renewcommand{\Gin@extensions}{.pdf,%
                    .png,.jpg,.bmp,.pict,.tif,.psd,.mac,.sga,.tga,.gif,%
                    .eps,.ps,%
                    }}{}
\makeatother

\ifboolexpr{bool{xetex} or bool{luatex}} % test for XeTeX/LuaTeX
 {}                                      % input encoding is utf8 by default
 {\usepackage[utf8]{inputenc}}           % switch to utf8

\usepackage[USenglish]{babel}


\ifboolexpr{bool{jacowbiblatex}}%        % if BibLaTeX is used
 {%
  \addbibresource{jacow-test.bib}
  \addbibresource{biblatex-examples.bib}
 }{}

\newcommand\SEC[1]{\textbf{\uppercase{#1}}}

%%
%%   Lengths for the spaces in the title
%%   \setlength\titleblockstartskip{..}  %before title, default 3pt
%%   \setlength\titleblockmiddleskip{..} %between title + author, default 1em
%%   \setlength\titleblockendskip{..}    %afterauthor, default 1em

%\copyrightspace %default 1cm. arbitrary size with e.g. \copyrightspace[2cm]

% testing to fill the copyright space
%\usepackage{eso-pic}
%\AddToShipoutPictureFG*{\AtTextLowerLeft{\textcolor{red}{COPYRIGHTSPACE}}}

\begin{document}

\title{Verification of the Neutron Mirror Capabilities in MCNPX via Gold Foil Measurements at the EIGER Instrument Beamline at the Swiss Spallation Neutron Source (SINQ) \thanks{Work supported by Swiss National Fund}}

\author{Ryan M. Bergmann \thanks{ryan.bergmann@psi.ch}, Vadim Talanov, Michael Wohlmuther, Tibor Reiss, Uwe Filges, \\ Sven Forss, Davide Reggiani, Uwe Stuhr, Emmanouela Rantsiou\\ Paul Scherrer Institut, Villigen, Switzerland\\}

\maketitle

%
\begin{abstract}
   The EIGER triple-axis thermal neutron spectrometer beamline contains “supermirror” neutron guides, which preferentially reflect low-energy neutrons toward the EIGER spectrometer that come from the ambient temperature, light water neutron source in SINQ.  Gold foil and neutron imaging plate measurements have been performed at the EIGER beamline in 2013.  This process can be modeled from incident proton to thermal neutron exiting the EIGER beamline by using the neutron mirror capabilities of MCNPX, which should be more accurate than simulations with simplified neutron source distributions and geometry representations.  The supermirror reflectivity parameters have been measured previously, and these parameters are used in MCNPX 2.7.0 to reproduce the activation measured from the gold foil irradiation, verifying the neutron mirror modeling capabilities in MCNPX 2.7.0.
\end{abstract}


\section{Introduction}

SINQ spallation
reflection

\begin{equation}\label{eq:units}
    C_B=\frac{q^3}{3\epsilon_{0} mc}=\SI{3.54}{\micro eV/T}
\end{equation}

eiger reflectivity parameters


\section{Measurement}

\begin{figure}[!htb]
   \centering
   \includegraphics*[width=45mm]{graphics/foils.eps}
   \caption{Gold foil array used in the measurements.  Foils are 25 mm in diameter and 30 mm center-to-center.  Blue dashed lines show the approximate extent of the neutron guide.}
   \label{foils}
\end{figure}

gold foils
current
activation

\section{MCNPX 2.7.0 Simulation}

\begin{figure}[!htb]
   \centering
   \includegraphics*[width=65mm]{graphics/geom.pdf}
   \caption{MCNPX geometry highlighting the important elements of the neutorn path to the measurement point.}
   \label{geom}
\end{figure}

\begin{figure}[!htb]
   \centering
   \includegraphics*[trim = 0mm 170mm 0mm 0mm, width=85mm]{graphics/ws.pdf}
   \caption{MCNPX weight window map near the water scatterer.}
   \label{wsmap}
\end{figure}

ww, dxtran
plot of geometry, wwg
ws and sapphire
reflective surfaces


\section{Results}

activity formula?

%trim option's parameter order: left bottom right top
\begin{figure}[!htb]
   \centering
   \includegraphics*[trim = 10mm 0mm 15mm 0mm, width=80mm]{graphics/gain.pdf}
   \caption{MCNPX weight window map near the water scatterer.}
   \label{gain}
\end{figure}

\begin{figure}[!htb]
   \centering
   \includegraphics*[trim = 0mm 0mm 15mm 0mm, width=80mm]{graphics/GF_act.pdf}
   \caption{MCNPX weight window map near the water scatterer.}
   \label{activation}
\end{figure}

gain goes above 1 at about 200 meV


\section{Conclusion}

30\% of ideal


\section{acknowledgment}

Work supported by Swiss National Fund



%\begin{thebibliography}{9}   % Use for  1-9  references
\begin{thebibliography}{99} % Use for 10-99 references

\bibitem{mcnpx-ref}
	F. X. Gallmeier, M. Wohlmuther, U. Filges, D. Kiselev, G. Muhrer,
	``Implementation of Neutron Mirror Modeling Capability into MCNPX and Its Demonstration in First Applications'',
  Nuclear Technology Vol. 168 No. 3, December 2009, ppg 768-772.

\bibitem{eiger-uwe}
  U. Filges, S. N. Gvasaliya, H.M. Ronnow, J. Birk,
  ``Monte Carlo Simulations for the new EIGER spectrometer at PSI'',
  4th European Conference on Neutron Scattering, 25-29 June, 2007, Lund, Sweden.

\end{thebibliography}


\end{document}
