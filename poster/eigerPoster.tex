\documentclass[final,t]{beamer}
\mode<presentation>
{
%  \usetheme{Warsaw}
%  \usetheme{Aachen}
%  \usetheme{Oldi6}
%  \usetheme{I6td}
  \usetheme{I6dv}
%  \usetheme{I6pd}
%  \usetheme{I6pd2}
}
% additional settings
\setbeamerfont{itemize}{size=\normalsize}
\setbeamerfont{itemize/enumerate body}{size=\normalsize}
\setbeamerfont{itemize/enumerate subbody}{size=\normalsize}

% additional packages
\usepackage{times}
\usepackage{amsmath,amsthm, amssymb, latexsym}
\usepackage{exscale}
%\boldmath
\usepackage{booktabs, array}
%\usepackage{rotating} %sideways environment
\usepackage[english]{babel}
\usepackage[latin1]{inputenc}
\usepackage[orientation=landscape,size=A0]{beamerposter}
\listfiles
\graphicspath{{../graphics/}}
\setbeamertemplate{bibliography item}[text]
% Display a grid to help align images
%\beamertemplategridbackground[1cm]
 
\title{\huge Verification of the Neutron Mirror Capabilities in MCNPX via Gold Foil Measurements at the EIGER Instrument Beamline at the Swiss Spallation Neutron Source (SINQ) } %\thanks{Supported by Swiss National Science Foundation grant 200021\_150048/1}
\author{R.M. Bergmann, U. Filges, S. Forss, D. Reggiani, E. Rantsiou, T. Reiss, U. Stuhr, V. Talanov, M. Wohlmuther} %\thanks{ryan.bergmann@psi.ch}
\institute[PSI]{Paul Scherrer Institut, Villigen, Switzerland}
\date[May 7th, 2015]{May 7th, 2015}

% abbreviations
\usepackage{xspace}
\makeatletter
\DeclareRobustCommand\onedot{\futurelet\@let@token\@onedot}
\def\@onedot{\ifx\@let@token.\else.\null\fi\xspace}
\def\eg{{e.g}\onedot} \def\Eg{{E.g}\onedot}
\def\ie{{i.e}\onedot} \def\Ie{{I.e}\onedot}
\def\cf{{c.f}\onedot} \def\Cf{{C.f}\onedot}
\def\etc{{etc}\onedot}
\def\vs{{vs}\onedot}
\def\wrt{w.r.t\onedot}
\def\dof{d.o.f\onedot}
\def\etal{{et al}\onedot}
\makeatother




%%%%%%%%%%%%%%%%%%%%%%%%%%%%%%%%%%%%%%%%%%%%%%%%%%%%%%%%%%%%%%%%%%%%%%%%%%%%%%%%%%%%%%%%%%%%%%%%%%%%%%%%%%%%
%%%%%%%%%%%%%%%%%%%%%%%%%%%%%%%%%%%%%%%%%%%%%%%%%%%%%%%%%%%%%%%%%%%%%%%%%%%%%%%%%%%%%%%%%%%%%%%%%%%%%%%%%%%%
\begin{document}
\begin{frame}{} 
  \begin{columns}[t]
    \begin{column}{.3\linewidth}

      %%%%%%%%%%%%%%%%%%%%%%%%%%%%%%%%%%%%%%%%%%%%%%%%%%%%%%%%%%%%%%%%%%%%%%%%%%%%%%%%%%%%%%%%%%%%%%%%%%%%%%%%%%%%

      \begin{block}{Introduction}

\begin{columns}
\begin{column}{.6\linewidth}
\noindent{\hskip1cm\textbf{SINQ}}
\begin{itemize}
  \item Spallation neutron source
  \item Continuous 590 MeV proton beam
  \item Lead target
  \item D$_2$O moderator tank  
\end{itemize}

\noindent{\hskip1cm\textbf{EIGER}}
\begin{itemize}
\item Triple-axis thermal neutron spectrometer 
\end{itemize}

\noindent{\hskip1cm\textbf{Light Water Scatterer}} 
\begin{itemize}
\item Seven vertical aluminum tubes near the target 
\end{itemize}

\noindent{\hskip1cm\textbf{Neutron Guide}} 
  \begin{itemize} 
  \item Supermirrors reflective to low-energy neutrons
  \item Allow the guide to transport neutrons with small divergence
  \item Reflectivity parameters of EIGER's supermirrors have been measured
  \item Sapphire crystal filters $>$0.1 eV \cite{freund}.
  \end{itemize}
\end{column}

\begin{column}{.4\linewidth}
\noindent{\hskip1cm\textbf{Study Goals}} 
  \begin{itemize}
  \item Verify neutron reflectivity is working in MCNPX 2.7.0
  \item Verify that reflectivity works with MPI parallelism
  \item Compare calculated and measured gold foil activities
  \end{itemize}

    \includegraphics*[width=\linewidth]{reflectivity_curve.pdf}

\end{column}
\end{columns}
\vspace{2ex}
\begin{equation}\label{eq:ref}
        R(Q) = 
        \begin{cases}
            Q > Q_c;  \qquad \frac{R_0}{2}\left\{  1 - \tanh\left(  \frac{Q - m Q_c}{W}\right) \right\}\{1-\alpha(Q-Q_c)\} \\
            Q \leq Q_c; \qquad R_0 \\
        \end{cases}
    \end{equation}

      \end{block}

      %%%%%%%%%%%%%%%%%%%%%%%%%%%%%%%%%%%%%%%%%%%%%%%%%%%%%%%%%%%%%%%%%%%%%%%%%%%%%%%%%%%%%%%%%%%%%%%%%%%%%%%%%%%%
      \begin{block}{EIGER Beamline Geometry}
      
        \begin{columns}[T]
          \begin{column}{.75\linewidth}
        
        \includegraphics*[width=\linewidth]{exp.jpg}

        \includegraphics*[width=\linewidth]{geom.pdf}
        
        \end{column}
          \begin{column}{.25\linewidth}

          \begin{center}
          \includegraphics*[width=\linewidth]{eiger_lr.png}
          \end{center}

          \begin{itemize} 
          \item As high-fidelity as possible
          \item S($\alpha$,$\beta$) data used, including those for the crystal sapphire \cite{sapp}
          \item Water scatterer model created from CAD models using MCAM \cite{mcam}
          \end{itemize}

        \end{column}
      \end{columns}
      
      \end{block}

      %%%%%%%%%%%%%%%%%%%%%%%%%%%%%%%%%%%%%%%%%%%%%%%%%%%%%%%%%%%%%%%%%%%%%%%%%%%%%%%%%%%%%%%%%%%%%%%%%%%%%%%%%%%%

      \begin{block}{Gold Foil Measurement}
        \begin{columns}[T]
          \begin{column}{.65\linewidth}
          \begin{itemize} 
          \item Vertical array of six gold foils, 25 mm in diameter, 30 mm center-to-center
          \item Placed near the monochromator position at EIGER
          \item Exposed to $200 \mu A$ for 5 minutes in December, 2013
          \item Activation analysis was performed using a gamma spectrometer after irradiation
          \end{itemize}

            
          \end{column}
          \begin{column}{.35\linewidth}
            \includegraphics*[angle=90,width=\linewidth]{foils.eps}
          \end{column}
        \end{columns}
      \end{block}


      %%%%%%%%%%%%%%%%%%%%%%%%%%%%%%%%%%%%%%%%%%%%%%%%%%%%%%%%%%%%%%%%%%%%%%%%%%%%%%%%%%%%%%%%%%%%%%%%%%%%%%%%%%%%

      \end{column}

      %%%%%%%%%%%%%%%%%%%%%%%%%%%%%%%%%%%%%%%%%%%%%%%%%%%%%%%%%%%%%%%%%%%%%%%%%%%%%%%%%%%%%%%%%%%%%%%%%%%%%%%%%%%%


    \begin{column}{.3\linewidth}

      \begin{block}{Variance Reduction}

      \begin{center} Approximately $\sim 10^{-6}$ neutrons per proton make it to EIGER. \end{center}

      \begin{columns}

      \begin{column}{0.5\linewidth}

      \begin{center}
      \includegraphics*[width=\linewidth]{wwg.pdf}

      \includegraphics*[width=\linewidth]{wwg_mesh.pdf}
      \end{center}

      \noindent{\hskip1cm\textbf{Mesh-Based Weight Windows}}
      \begin{itemize}
      \item Spherical mesh centered around the water scatterer 
      \item Fine divisions at small radii and in the direction of the target 
      \item Optimized on response of a point detector at the measurement position
      \end{itemize}

      \end{column}


      \begin{column}{0.5\linewidth}

      \noindent{\hskip1cm\textbf{DXTRAN Sphere}}
      \begin{itemize}
      \item Placed around the sapphire crystal 
      \item Force a fraction of a neutron's weight to be placed at the sphere radius at every collision.
      \item ``Uncollided'' neutrons transported normally after placement \cite{mcnpx270}
      \end{itemize}

      \begin{center}
      \includegraphics*[width=.85\linewidth]{crystal.pdf}
      \end{center}
      
      \end{column}

      \end{columns}

      \end{block}

      \begin{block}{Intermediate Results}

     \includegraphics*[width=\linewidth]{dists.pdf}

      \begin{columns}

      \begin{column}{0.5\linewidth}

       

        \includegraphics*[width=\linewidth]{specs_narrow.eps}

      \end{column}

      \begin{column}{0.5\linewidth}

      \begin{itemize}
        \item Visual confirmation of ditribution convergence
        \item The post-sapphire spectrum has a proportionally smaller component above 0.1 eV
        \item The post-sapphire distribution used as the source in subsequent calculations with the neutron guide reflectivity turned on and off. 
        \end{itemize}

      \end{column}

      \end{columns}

      \end{block}

      %%%%%%%%%%%%%%%%%%%%%%%%%%%%%%%%%%%%%%%%%%%%%%%%%%%%%%%%%%%%%%%%%%%%%%%%%%%%%%%%%%%%%%%%%%%%%%%%%%%%%%%%%%%%


    \end{column}

    %%%%%%%%%%%%%%%%%%%%%%%%%%%%%%%%%%%%%%%%%%%%%%%%%%%%%%%%%%%%%%%%%%%%%%%%%%%%%%%%%%%%%%%%%%%%%%%%%%%%
    
    \begin{column}{.3\linewidth}

      \begin{block}{Experimental \vs Calculated Results}

      %\begin{columns}

      %\begin{column}{0.5\linewidth}

      \begin{center}
      \includegraphics*[width=0.75\linewidth]{gain.pdf}
      \end{center}

      \begin{itemize}
      \item Surface source sampled repeatedly with different seeds  
      %\item 256 equi-log bins from $10^{-12}$ to 600 MeV
      \item Reflectivity $>$1 below 200 meV and increases to 2.5 at 3 meV
      \item The gold (n,$\gamma$) folded to determine the total reaction rate per incident proton
      \end{itemize}

      %\end{column}


      %\begin{column}{0.5\linewidth}

      \begin{center}
      \includegraphics*[width=0.75\linewidth]{GF_act.pdf}
      \end{center}

      \begin{itemize}
      \item Peak at center, less than 50\% of ideal
      \item Equal to the non-reflective guide edges
      \item The average value is 20\% if the -7.5 cm value is not considered (potential mass error)
      \end{itemize}
      
      %\end{column}

      %\end{columns}

      \end{block}

      %%%%%%%%%%%%%%%%%%%%%%%%%%%%%%%%%%%%%%%%%%%%%%%%%%%%%%%%%%%%%%%%%%%%%%%%%%%%%%%%%%%%%%%%%%%%%%%%%%%

      %\vspace{5ex}

      \begin{block}{\alert{Conclusions}}
        \begin{itemize}
        \item Confirmation guide reflectivity can be completely modeled in MCNPX 2.7.0
        \item Results indicate EIGER supermirrors are operating at 20\% of ``ideal'' capacity 
        \end{itemize}
        \vspace{-1ex}
      \end{block}

      %\vspace{5ex}

      %%%%%%%%%%%%%%%%%%%%%%%%%%%%%%%%%%%%%%%%%%%%%%%%%%%%%%%%%%%%%%%%%%%%%%%%%%%%%%%%%%%%%%%%%%%%%%%%%%%

            \begin{block}{References}

      \bibliographystyle{unsrt}
        \begin{thebibliography}{9}   % Use for  1-9  references
\scriptsize
\bibitem{mcnpx-ref}
  F. X. Gallmeier, M. Wohlmuther, U. Filges, D. Kiselev, G. Muhrer,
  ``Implementation of Neutron Mirror Modeling Capability into MCNPX and Its Demonstration in First Applications'',
  Nuclear Technology Vol. 168 No. 3, December 2009, pp 768-772. 

\bibitem{eiger-uwe}
  U. Filges, S. N. Gvasaliya, H.M. Ronnow, J. Birk,
  ``Monte Carlo Simulations for the new EIGER spectrometer at PSI'',
  4th European Conference on Neutron Scattering, 25-29 June, 2007, Lund, Sweden. 

\bibitem{freund}
A. K. Freund,
``Cross-sections of materials used as neutron monochromators and filters'',
Nuclear Instruments and Methods in Physics Research, vol 213, 1983, pp 495-501.

\bibitem{mcnpx270}
   D.B. Pelowitz,
   ``MCNPX Manual, Version 2.7.0'',
   La-Cp-11-00438, April, 2011.

\bibitem{sapp}
  F. Cantargi, J.R. Granada, R.E. Mayer, 
  ``Thermal neutron scattering kernels for sapphire and silicon single crystals'', 
  Annals of Nuclear Energy 80, 2015, pp 43–46.  

\bibitem{mcam}
  Y. Wu, FDS Team, 
  ``CAD-based interface programs for fusion neutron transport simulation'', 
  Fusion Engineering and Design 84, 2009, 1987–1992. \\

\end{thebibliography}
      \end{block}

%%%%%%%%%%%%%%%%%%%%%%%%%%%%%%%%%%%%%%%%%%%%%%%%%%%%%%%

    \end{column}
  \end{columns}



\end{frame}

\end{document}


%%%%%%%%%%%%%%%%%%%%%%%%%%%%%%%%%%%%%%%%%%%%%%%%%%%%%%%%%%%%%%%%%%%%%%%%%%%%%%%%%%%%%%%%%%%%%%%%%%%%
%%% Local Variables: 
%%% mode: latex
%%% TeX-PDF-mode: t
%%% End: 