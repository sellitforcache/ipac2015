\documentclass[final,t]{beamer}
\mode<presentation>
{
%  \usetheme{Warsaw}
%  \usetheme{Aachen}
%  \usetheme{Oldi6}
%  \usetheme{I6td}
  \usetheme{I6dv}
%  \usetheme{I6pd}
%  \usetheme{I6pd2}
}
% additional settings
\setbeamerfont{itemize}{size=\normalsize}
\setbeamerfont{itemize/enumerate body}{size=\normalsize}
\setbeamerfont{itemize/enumerate subbody}{size=\normalsize}

% additional packages
\usepackage{times}
\usepackage{amsmath,amsthm, amssymb, latexsym}
\usepackage{exscale}
%\boldmath
\usepackage{booktabs, array}
%\usepackage{rotating} %sideways environment
\usepackage[english]{babel}
\usepackage[latin1]{inputenc}
\usepackage[orientation=landscape,size=A0]{beamerposter}
\listfiles
\graphicspath{{../graphics/}}
\setbeamertemplate{bibliography item}[text]
% Display a grid to help align images
%\beamertemplategridbackground[1cm]
 
\title{\huge Verification of the Neutron Mirror Capabilities in MCNPX via Gold Foil Measurements at the EIGER Instrument Beamline at the Swiss Spallation Neutron Source (SINQ) } %\thanks{Supported by Swiss National Science Foundation grant 200021\_150048/1}
\author{R.M. Bergmann, U. Filges, S. Forss, D. Reggiani, E. Rantsiou, T. Reiss, U. Stuhr, V. Talanov, M. Wohlmuther} %\thanks{ryan.bergmann@psi.ch}
\institute[PSI]{Paul Scherrer Institut, Villigen, Switzerland}
\date[May 7th, 2015]{May 7th, 2015}

% abbreviations
\usepackage{xspace}
\makeatletter
\DeclareRobustCommand\onedot{\futurelet\@let@token\@onedot}
\def\@onedot{\ifx\@let@token.\else.\null\fi\xspace}
\def\eg{{e.g}\onedot} \def\Eg{{E.g}\onedot}
\def\ie{{i.e}\onedot} \def\Ie{{I.e}\onedot}
\def\cf{{c.f}\onedot} \def\Cf{{C.f}\onedot}
\def\etc{{etc}\onedot}
\def\vs{{vs}\onedot}
\def\wrt{w.r.t\onedot}
\def\dof{d.o.f\onedot}
\def\etal{{et al}\onedot}
\makeatother




%%%%%%%%%%%%%%%%%%%%%%%%%%%%%%%%%%%%%%%%%%%%%%%%%%%%%%%%%%%%%%%%%%%%%%%%%%%%%%%%%%%%%%%%%%%%%%%%%%%%%%%%%%%%
%%%%%%%%%%%%%%%%%%%%%%%%%%%%%%%%%%%%%%%%%%%%%%%%%%%%%%%%%%%%%%%%%%%%%%%%%%%%%%%%%%%%%%%%%%%%%%%%%%%%%%%%%%%%
\begin{document}
\begin{frame}{} 

\begin{columns}[t]

\begin{column}{.96\linewidth}
\begin{block}{Abstract}

The EIGER triple-axis thermal neutron spectrometer beamline contains ``supermirror'' neutron guides, which preferentially reflect low-energy neutrons toward the EIGER spectrometer that come from the ambient temperature, light water neutron source in SINQ.  Gold foil measurements have been performed at the EIGER beamline in 2013.  This process can be modeled from incident proton to thermal neutron exiting the EIGER beamline by using the neutron mirror capabilities of MCNPX, which should be more accurate than simulations with simplified neutron source distributions and geometry representations.  The supermirror reflectivity parameters have been measured previously and are used in MCNPX 2.7.0 to reproduce the activity measured from the gold foil irradiation, verifying the neutron mirror modeling capabilities in MCNPX 2.7.0.

\end{block}
\end{column}
\end{columns}

\vspace{2ex}

  \begin{columns}[t]
    \begin{column}{.3\linewidth}

      %%%%%%%%%%%%%%%%%%%%%%%%%%%%%%%%%%%%%%%%%%%%%%%%%%%%%%%%%%%%%%%%%%%%%%%%%%%%%%%%%%%%%%%%%%%%%%%%%%%%%%%%%%%%

      \begin{block}{Introduction}

\begin{columns}
\begin{column}{.5\linewidth}
\noindent{\hskip1cm\textbf{SINQ}}
\begin{itemize}
  \item Spallation neutron source
  \item Continuous 590 MeV proton beam
  \item Lead target
  \item D$_2$O moderator tank  
\end{itemize}

\noindent{\hskip1cm\textbf{EIGER}}
\begin{itemize}
\item Triple-axis thermal neutron spectrometer 
\end{itemize}

\noindent{\hskip1cm\textbf{Light Water Scatterer}} 
\begin{itemize}
\item Seven vertical aluminum tubes near the target 
\end{itemize}

\noindent{\hskip1cm\textbf{Neutron Guide}} 
  \begin{itemize} 
  \item Supermirrors reflective to low-energy neutrons \cite{mcnpx-ref}
  \item Allow the guide to transport neutrons with small divergence
  \item Reflectivity parameters of EIGER's supermirrors have been measured \cite{eiger-uwe}
  \item Sapphire crystal filter scatters out neutrons $>$0.1 eV \cite{freund}
  \end{itemize}
\end{column}

\begin{column}{.5\linewidth}
\noindent{\hskip1cm\textbf{Study Goals}} 
  \begin{itemize}
  \item Verify neutron reflectivity is working in MCNPX 2.7.0 \cite{mcnpx-ref,mcnpx270}
  \item Verify that reflectivity works with MPI parallelism
  \item Compare calculated and measured gold foil activities
  \end{itemize}

  \noindent{\hskip1cm\textbf{EIGER Parameters}}  
        \begin{tabular}{l l l l}
        $R_0$    & = $0.995           $ & $ Q_c  $  & = $ 2.17 \times 10^{-2} $\\
        $m  $    & = $3.6             $ & $\alpha$  & = $ 3.99 $ \\
        $W  $    & = $1 \times 10^{-3}$ & $      $  &   
        \end{tabular}

    \includegraphics*[width=\linewidth]{reflectivity_curve.pdf}

\end{column}
\end{columns}
\vspace{-4ex}
\noindent{\hskip1cm\textbf{Supermirror Reflectivity }}
\begin{center}
\begin{equation*}\label{eq:ref}
        R(Q) = 
        \begin{cases}
            Q > Q_c;  \qquad \frac{R_0}{2}\left\{  1 - \tanh\left(  \frac{Q - m Q_c}{W}\right) \right\}\{1-\alpha(Q-Q_c)\} \\
            Q \leq Q_c; \qquad R_0 \\
        \end{cases}
    \end{equation*}

        \begin{tabular}{l l l l}
        $R_0$    & = nominal reflectivity & $ Q_c  $  & = critical momentum transfer   \\
        $m  $    & = angular extension    & $\alpha$  & = reflectivity declination  \\
        $W  $    & = edge width           &           &   
        \end{tabular}
\end{center} 
      \end{block}

      %%%%%%%%%%%%%%%%%%%%%%%%%%%%%%%%%%%%%%%%%%%%%%%%%%%%%%%%%%%%%%%%%%%%%%%%%%%%%%%%%%%%%%%%%%%%%%%%%%%%%%%%%%%%
      \vspace{3ex}
      \begin{block}{EIGER Beamline Geometry}
      
        \begin{columns}[T]
          \begin{column}{.75\linewidth}
        
        \includegraphics*[width=\linewidth]{exp.jpg}

        \includegraphics*[width=\linewidth]{geom.pdf}
        
        \end{column}
          \begin{column}{.25\linewidth}

          \begin{center}
          \includegraphics*[width=\linewidth]{eiger_lr.png}
          \end{center}

          \noindent{\hskip1cm\textbf{MCNPX Model}} 
          \begin{itemize} 
          \item As high-fidelity as possible
          \item S($\alpha$,$\beta$) data used, including those for the sapphire crystal \cite{sapp}
          \item Water scatterer model created from CAD models using MCAM \cite{mcam}
          \end{itemize}

          \begin{center}
          \includegraphics*[width=.5\linewidth]{wsgeom_xy.pdf} \quad \includegraphics*[width=.31\linewidth]{wsgeom_yz.pdf} 
          \end{center}

        \end{column}
      \end{columns}
      
      \end{block}

      %%%%%%%%%%%%%%%%%%%%%%%%%%%%%%%%%%%%%%%%%%%%%%%%%%%%%%%%%%%%%%%%%%%%%%%%%%%%%%%%%%%%%%%%%%%%%%%%%%%%%%%%%%%%

      \end{column}

      %%%%%%%%%%%%%%%%%%%%%%%%%%%%%%%%%%%%%%%%%%%%%%%%%%%%%%%%%%%%%%%%%%%%%%%%%%%%%%%%%%%%%%%%%%%%%%%%%%%%%%%%%%%%


    \begin{column}{.3\linewidth}

      \begin{block}{Variance Reduction}
      \vspace{-2ex}
      \begin{center} Approximately $10^{-6}$ neutrons per proton make it to EIGER. \end{center}

      \begin{columns}

      \begin{column}{0.5\linewidth}

      \begin{center}
      \includegraphics*[width=.15\linewidth]{wwg_key.pdf} \includegraphics*[width=.85\linewidth]{wwg.pdf}

      \includegraphics*[width=\linewidth]{wwg_mesh.pdf}
      \end{center}

      \noindent{\hskip1cm\textbf{Mesh-Based Weight Windows}}
      \begin{itemize}
      \item Spherical mesh centered around the water scatterer 
      \item Fine divisions at small radii and in the direction of the target 
      \item Optimized on response of a point detector at the measurement position
      \end{itemize}

      \end{column}


      \begin{column}{0.5\linewidth}

      \noindent{\hskip1cm\textbf{DXTRAN Sphere}}
      \begin{itemize}
      \item Placed around the sapphire crystal 
      \item Force a fraction of a neutron's weight to be placed at the sphere radius at every collision.
      \item ``Uncollided'' neutrons transported normally after placement \cite{mcnpx270}
      \end{itemize}

      \begin{center}
      \includegraphics*[width=.85\linewidth]{crystal.pdf}
      \end{center}
      
      \end{column}

      \end{columns}

      \end{block}

      \begin{block}{Intermediate Results}
      %\vspace{-2ex}
      \noindent{\qquad \qquad \textbf{Water Scatterer} {\tiny (Black = Guide view, Pink = Nozzle view)} \qquad \qquad  \textbf{Sapphire Crystal}  }
      \begin{center}
      \includegraphics*[width=\linewidth]{dists.pdf}
      \end{center}

      \begin{columns}

      \begin{column}{0.5\linewidth}
      \vspace{-2ex}
        \includegraphics*[width=\linewidth]{specs_narrow.eps}
      \end{column}
      \begin{column}{0.5\linewidth}
      \vspace{-2ex}
      \begin{itemize}
        \item Visual confirmation of distribution convergence
        \item The post-sapphire spectrum has a proportionally smaller component above 0.1 eV
        \item The post-sapphire distribution used as the source in subsequent calculations with the neutron guide reflectivity turned on and off. 
        \end{itemize}
      \end{column}

      \end{columns}

      \end{block}

      %%%%%%%%%%%%%%%%%%%%%%%%%%%%%%%%%%%%%%%%%%%%%%%%%%%%%%%%%%%%%%%%%%%%%%%%%%%%%%%%%%%%%%%%%%%%%%%%%%%%%%%%%%%%


    \end{column}

    %%%%%%%%%%%%%%%%%%%%%%%%%%%%%%%%%%%%%%%%%%%%%%%%%%%%%%%%%%%%%%%%%%%%%%%%%%%%%%%%%%%%%%%%%%%%%%%%%%%%
    
    \begin{column}{.3\linewidth}

      \begin{block}{Gold Foil Measurement}
        \begin{columns}[T]
          \begin{column}{.65\linewidth}
          \begin{itemize} 
          \item Vertical array of six gold foils, \alert{25 mm} in diameter, 30 mm center-to-center
          \item Placed near the monochromator position at EIGER
          \item Exposed to \alert{$200 \mu A$} for \alert{5 minutes} in December, 2013
          \item Activation analysis was performed using a gamma spectrometer after irradiation
          \end{itemize}

            
          \end{column}
          \begin{column}{.35\linewidth}
            \includegraphics*[angle=90,width=\linewidth]{foils.eps}
          \end{column}
        \end{columns}
      \end{block}

      %%%%%%%%%%%%%%%%%%%%%%%%%%%%%%%%%%%%%%%%%%%%%%%%%%%%%%%%%%%%%%%%%%%%%%%%%%%%%%%%%%%%%%%%%%%%%%%%%%%%%%%%%%%%
      \vspace{8ex}
      \begin{block}{Experimental \vs Calculated Results}

      \begin{columns}

      \begin{column}{0.5\linewidth}
      \vspace{-4ex}
      \begin{center}
      \noindent{\hskip1cm\textbf{Spectra}}
      \includegraphics*[width=\linewidth]{gain.pdf}
      \end{center}
      \end{column}
      \begin{column}{0.5\linewidth}
      \vspace{-4ex}
      \begin{center}
      \noindent{\hskip1cm\textbf{Activities}}
      \includegraphics*[width=\linewidth]{GF_act.pdf}
      \end{center}
      \end{column}

      \end{columns}




      \begin{columns}

      \begin{column}{0.5\linewidth}

      \begin{itemize}
      \item Surface source sampled repeatedly with different seeds  
      \item 256 equi-log bins from $10^{-12}$ to 600 MeV
      \item Reflectivity gain \alert{$>$1} below 200 meV 
      \item Reflectivity gain increases to \alert{2.5} at 3 meV
      \item The gold (n,$\gamma$) folded to determine the total reaction rate per incident proton
      \end{itemize}

      \end{column}

      \begin{column}{0.5\linewidth}

      \begin{itemize}
      \item Peak at vertical center, less than \alert{50\%} of ideal
      \item Equal to a non-reflective guide at vertical edges
      \item The average value is \alert{20\%} if the -7.5 cm value is not considered (potential mass error)
      \end{itemize}

      \end{column}

      \end{columns}
      
      

      \end{block}

      %%%%%%%%%%%%%%%%%%%%%%%%%%%%%%%%%%%%%%%%%%%%%%%%%%%%%%%%%%%%%%%%%%%%%%%%%%%%%%%%%%%%%%%%%%%%%%%%%%%

      \vspace{8ex}

      \begin{block}{\alert{Conclusions}}
        \begin{itemize}
        \item Confirmation that guide reflectivity can be completely modeled in MCNPX 2.7.0
        \item Results indicate EIGER neutron supermirrors are operating at 20\% of ``ideal'' capacity 
        \end{itemize}
        \vspace{-1ex}
      \end{block}

      \vspace{8ex}

      %%%%%%%%%%%%%%%%%%%%%%%%%%%%%%%%%%%%%%%%%%%%%%%%%%%%%%%%%%%%%%%%%%%%%%%%%%%%%%%%%%%%%%%%%%%%%%%%%%%


      \begin{block}{References}

      \bibliographystyle{unsrt}
        \begin{thebibliography}{9}   % Use for  1-9  references
\scriptsize
\bibitem{mcnpx-ref}
  F. X. Gallmeier, M. Wohlmuther, U. Filges, D. Kiselev, G. Muhrer,
  ``Implementation of Neutron Mirror Modeling Capability into MCNPX and Its Demonstration in First Applications'',
  Nuclear Technology Vol. 168 No. 3, December 2009, pp 768-772. 

\bibitem{eiger-uwe}
  U. Filges, S. N. Gvasaliya, H.M. Ronnow, J. Birk,
  ``Monte Carlo Simulations for the new EIGER spectrometer at PSI'',
  4th European Conference on Neutron Scattering, 25-29 June, 2007, Lund, Sweden. 

\bibitem{freund}
A. K. Freund,
``Cross-sections of materials used as neutron monochromators and filters'',
Nuclear Instruments and Methods in Physics Research, vol 213, 1983, pp 495-501.

\bibitem{mcnpx270}
   D.B. Pelowitz,
   ``MCNPX Manual, Version 2.7.0'',
   La-Cp-11-00438, April, 2011.

\bibitem{sapp}
  F. Cantargi, J.R. Granada, R.E. Mayer, 
  ``Thermal neutron scattering kernels for sapphire and silicon single crystals'', 
  Annals of Nuclear Energy 80, 2015, pp 43-46.  

\bibitem{mcam}
  Y. Wu, FDS Team, 
  ``CAD-based interface programs for fusion neutron transport simulation'', 
  Fusion Engineering and Design 84, 2009, 1987-1992. \\

\end{thebibliography}
      \end{block}

%%%%%%%%%%%%%%%%%%%%%%%%%%%%%%%%%%%%%%%%%%%%%%%%%%%%%%%

    \end{column}
  \end{columns}



\end{frame}

\end{document}


%%%%%%%%%%%%%%%%%%%%%%%%%%%%%%%%%%%%%%%%%%%%%%%%%%%%%%%%%%%%%%%%%%%%%%%%%%%%%%%%%%%%%%%%%%%%%%%%%%%%
%%% Local Variables: 
%%% mode: latex
%%% TeX-PDF-mode: t
%%% End: 